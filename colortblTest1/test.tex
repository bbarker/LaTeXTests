\documentclass[compress,red,notes]{beamer}
%
% Discussion at:
% http://tex.stackexchange.com/questions/181330/when-using-colortbl-column-is-still-white-what-am-i-doing-wrong
%
\usepackage{tabularx}
\usepackage{xcolor}
\usepackage{colortbl}

\definecolor{Green}{rgb}{0,1,0}

\begin{document}

\frame{\frametitle{Performance} 

\begin{figure}
\centering
\resizebox{\textwidth}{!}{%
\begin{tabular}{l*{7}{r}>{\columncolor{Green}}r}
\textbf{(a)} \hspace{1.2cm} & Max. $\mu$ & Model & 
  Standard FBA & Fitted FBA & GIMME & iMAT & Lee et al. & FALCON\\
 & 75 \%& Yeast~5 MC & 0.66 & 0.66 & NaN  & 0.57 & 0.64 & 1 \\
 & 75 \%& Yeast~7 MC & 0.66 & 0.66 & 0.68 & 0.66 & 0.66 & 0.98\\
 & 75 \%& Yeast~5 HC & 0.73 & 0.78 & 0.75 & 0.66 & 0.98 & 0.99\\
Pearson's r  
 & 75 \%& Yeast~7 HC & 0.70 & 0.70 & 0.80 & 0.66 & 0.98 & 0.99\\
 & 85 \%& Yeast~7 MC & 0.62 & 0.62 & 0.65 & 0.62 & 0.62 & 0.97\\
 & 85 \%& Yeast~5 HC & 0.88 & 0.89 & 0.9  & 0.81 & 0.99 & 0.99\\
 & 85 \%& Yeast~7 HC & 0.67 & 0.67 & 0.87 & 0.62 & 0.98 & 0.98\\
\end{tabular}}
\caption{Here is a caption.}
\label{tab:FalcPerf}
\end{figure}
}

\end{document}